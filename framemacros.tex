\newcommand{\frametoc}{
	{
		\setbeamertemplate{footline}{}%remove footline
		\begin{frame}
			\frametitle{Table of Contents}
			\tableofcontents[hideallsubsections]%only sections are visibe; change if needed
		\end{frame}
	}
}

\newcommand{\frametitlepage}
{%-----------------------------------------------------------------------
% setting the layout for titlepage
%
{%
\setbeamertemplate{footline}[default]%
\setbeamertemplate{background}{%
  \vbox to \paperheight{\vfil\hbox to
\paperwidth{\hfil\ifpdf\includegraphics[clip, trim =  0mm 0mm  0mm 75mm,width=\paperwidth]{images/arr-plan-web.jpg}
\else\ifxetex\centering\includegraphics[clip, trim =  0mm 0mm  0mm 75mm,width=\paperwidth]{images/arr-plan-web.jpg}\else
\includegraphics[width=\paperwidth,natwidth=1261bp,natheight=668bp]{images/frontpage-bgp.pdf}%only needed if you use latex+dvipdf combo, which is strange in the first place; this option is relatively untested, use with discretion
  \fi
\fi
\hfil}}
}
%-----------------------------------------------------------------------%
\begin{frame}
  \titlepage %title/author(s),date, etc
  % \vspace{50pt}% uncomment this if need to play with the positioning of the text
	\includegraphics[width=45mm]{images/logo-ecole-polytechnique-right.png}% We use the converted png file instead of eps file provided by Ecole, because the eps file breaks the compilation; possible reason - some strange editing software used by Ecole. 
	\hfill\includegraphics[width=45mm]{images/lablogoalpha.png}% Logo of the lab as png with alpha-channel. Feel free to use your own image
  % \end{backgroundblock}
\end{frame}%
}%
}%